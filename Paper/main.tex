\documentclass[conference]{IEEEtran}
\IEEEoverridecommandlockouts
% The preceding line is only needed to identify funding in the first footnote. If that is unneeded, please comment it out.
\usepackage{cite}
\usepackage{amsmath,amssymb,amsfonts}
\usepackage{algorithmic}
\usepackage{graphicx}
\usepackage{textcomp}
\usepackage{xcolor}
\def\BibTeX{{\rm B\kern-.05em{\sc i\kern-.025em b}\kern-.08em
    T\kern-.1667em\lower.7ex\hbox{E}\kern-.125emX}}
    
\newcommand{\biburl}[2]{\url{#1} [Accessed: Jan 1, 2020]}
\newcommand{\itembf}[1]{\item \textbf{#1}}
% fix long vspaces in bibliography
\def\IEEEbibitemsep{0pt plus .5pt}    

\begin{document}

\title{Activity Recognition Through Converted Acceleration Data Into Images\\
%{\footnotesize \textsuperscript{*}Note: Sub-titles are not captured in Xplore and
%should not be used}
%\thanks{Identify applicable funding agency here. If none, delete this.}
}

\author{\IEEEauthorblockN{1\textsuperscript{st} Given Orhan Konak}
\IEEEauthorblockA{\textit{Digital Health Center} \\
\textit{Hasso-Plattner-Institute}\\
Potsdam, Germany \\
Orhan.Konak@hpi.de}
\and
\IEEEauthorblockN{2\textsuperscript{nd} Given Pit Wegner}
\IEEEauthorblockA{\textit{Digital Health Center} \\
\textit{Hasso-Plattner-Institute}\\
Potsdam, Germany \\
Pit.Wegner@student.hpi.uni-potsdam.de}
%\and
%\IEEEauthorblockN{3\textsuperscript{rd} Given Name Surname}
%\IEEEauthorblockA{\textit{dept. name of organization (of Aff.)} \\
%\textit{name of organization (of Aff.)}\\
%City, Country \\
%email address}
%\and
%\IEEEauthorblockN{4\textsuperscript{th} Given Name Surname}
%\IEEEauthorblockA{\textit{dept. name of organization (of Aff.)} \\
%\textit{name of organization (of Aff.)}\\
%City, Country \\
%email address}
%\and
%\IEEEauthorblockN{5\textsuperscript{th} Given Name Surname}
%\IEEEauthorblockA{\textit{dept. name of organization (of Aff.)} \\
%\textit{name of organization (of Aff.)}\\
%City, Country \\
%email address}
\and
\IEEEauthorblockN{3\textsuperscript{th} Prof. Dr. Bert Arnrich}
\IEEEauthorblockA{\textit{Digital Health Center} \\
\textit{Hasso-Plattner-Institute}\\
Potsdam, Germany \\
Bert.Arnrich@hpi.de}
}

\maketitle

\begin{abstract}
This document is a model and instructions for \LaTeX.
This and the IEEEtran.cls file define the components of your paper [title, text, heads, etc.]. *CRITICAL: Do Not Use Symbols, Special Characters, Footnotes, 
or Math in Paper Title or Abstract.
\end{abstract}

\begin{IEEEkeywords}
component, formatting, style, styling, insert
\end{IEEEkeywords}

\section{Introduction}



\section{Related Work}



\section{Methods}



\subsection{Data Transformation}\label{DT}



\subsection{Dimensionality Reduction}\label{DR}



\subsection{Windowing}\label{Windowing}

Punctuate equations with commas or periods when they are part of a sentence, as in:

\begin{equation}
a+b=\gamma\label{eq}
\end{equation}

Use ``\eqref{eq}'', not ``Eq.~\eqref{eq}'' or ``equation \eqref{eq}'', except at 
the beginning of a sentence: ``Equation \eqref{eq} is . . .''

\subsection{Image Classification}\label{ImgCl}



\subsection{Sequencing}\label{SQC}

\begin{table}[htbp]
\caption{Table Type Styles}
\begin{center}
\begin{tabular}{|c|c|c|c|}
\hline
\textbf{Table}&\multicolumn{3}{|c|}{\textbf{Table Column Head}} \\
\cline{2-4} 
\textbf{Head} & \textbf{\textit{Table column subhead}}& \textbf{\textit{Subhead}}& \textbf{\textit{Subhead}} \\
\hline
copy& More table copy$^{\mathrm{a}}$& &  \\
\hline
\multicolumn{4}{l}{$^{\mathrm{a}}$Sample of a Table footnote.}
\end{tabular}
\label{tab1}
\end{center}
\end{table}

%\begin{figure}[htbp]
%\centerline{\includegraphics{fig1.png}}
%\caption{Example of a figure caption.}
%\label{fig}
%\end{figure}

\section{Contributions}



\section{Evaluation and Discussion}



\section{Conclusion and Future Work}



\section*{Acknowledgment}

The preferred spelling of the word ``acknowledgment'' in America is without 
an ``e'' after the ``g''. Avoid the stilted expression ``one of us (R. B. 
G.) thanks $\ldots$''. Instead, try ``R. B. G. thanks$\ldots$''. Put sponsor 
acknowledgments in the unnumbered footnote on the first page.

\bibliography{refs}
\bibliographystyle{IEEEtran}

\end{document}
